\chapter{Introducci�n}

Se ha desarrollado un simulador de procedimientos para helic�pteros de cuatro
palas, un solo rotor principal y un rotor de cola. El simulador utiliza un 
modelo de 6 grados de libertad de s�lido r�gido para el fuselaje. Para el rotor 
calcula la soluci�n del grado de libertad de batimiento de cada pala y 3 grados 
de libertad para la velocidad inducida. Por �ltimo, incluye un grado de libertad 
para la velocidad de giro del rotor y el par del motor. Dados los cuatro controles
que ejerce el piloto y los datos atmosf�ricos puede calcular la respuesta del 
helic�ptero en cualquier situaci�n de vuelo. 

Los par�metros del helic�ptero se especifican en un fichero aparte. Se incluyen 
dos ficheros de ejemplo para el Lynx y el BlackHawk.

Por otra parte el simulador tiene las siguientes limitaciones:
\begin{itemize}
    \item No simula correctamente los l�mites de operaci�n del helic�ptero. 
    No predice la entrada en p�rdida del rotor. 
    \item No simula correctamente maniobras bruscas. 
    \item Sistema de control demasiado simple. Sistemas reales llenos de 
    no linealidades y mucho mas complejos.
    \item No tiene en cuenta efectos de interacci�n entre rotor, rotor de cola, 
    estabilizadores y fuselaje.
    \item No calcula el consumo de combustible y la variaci�n de la masa y los
    momentos de inercia debido a ello.
\end{itemize}

Adicionalmente se incluye c�digo, que se puede utilizar desde otros programas, 
o desde la consola interactiva de python para el c�lculo no interactivo
de la respuesta, el c�lculo de derivadas de estabilidad y el c�lculo del 
trimado del helic�ptero para diversas condiciones de vuelo. �ste �ltimo con la 
limitaci�n de que no converge bien para las situaciones con resbalamiento.
